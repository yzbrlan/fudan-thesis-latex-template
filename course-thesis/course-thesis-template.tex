\documentclass[UTF8,a4paper,12pt]{ctexart}%需要UTF8编码

\bibliographystyle{gbt7714-2005} %参考文献格式设为GBT7714-2005N.bst
\newcommand{\upcite}[1]{\textsuperscript{\textsuperscript{\cite{#1}}}}
\usepackage{url}

\usepackage[draft=false,colorlinks=true,CJKbookmarks=true,linkcolor=black,citecolor=green,urlcolor=blue]{hyperref} %参考文献跳转,此宏包会自动载入graphicx

\usepackage[top=3cm,bottom=2cm,left=2cm,right=2cm]{geometry} % 页边距

\usepackage{longtable}%生成跨页的表格
\usepackage{array} % for extrarowheight
\setlength{\extrarowheight}{1.5pt}


\usepackage[american]{babel}%它的引用是遇单词换行的时候,确保单词的切割是按照音节来而不是随意切割。这会让作为native speaker的审稿人赏心悦目,心中暗爽。
\usepackage{microtype}%它的最大特点就是能够调整全篇文章(或局部)的字间距,字间距最大调整范围为±1em。可使得某段落不会出现单独一个单词占一行,或文章末尾单独一行文字占一页的不美观情况(注,该包在NIPS中自带引用;而ECCV由于LNCS在排版方面的一些限制因素,不推荐在ECCV中使用该包引用);
\addto\captionsamerican{
	\renewcommand\contentsname{目录} 
	\renewcommand\listfigurename{插图目录} 
	\renewcommand\listtablename{表格目录} 
	\renewcommand\refname{参考文献} 
	\renewcommand\indexname{索引} 
	\renewcommand\figurename{图} 
	\renewcommand\tablename{表} 
	\renewcommand\abstractname{摘要} 
	\renewcommand\partname{部分} 
	\renewcommand\appendixname{附录} 
	\renewcommand\today{\number\year年\number\month月\number\day日} 
}

\setCJKmainfont{STKaitiSC-Regular} %中文字体

\usepackage{pdfpages}%插入封面

\usepackage{latexsym}

\usepackage{fancyhdr}
\pagestyle{fancy}
\lhead{\bfseries \leftmark}
\chead{}
\rhead{\thepage}
%\cfoot{姓名 16300000001}
\lfoot{}
\rfoot{}
\renewcommand{\headrulewidth}{0.4pt}
%\renewcommand{\footrulewidth}{0.4pt}
\title{论文模板}
\author{姓名 16300000001}
\date{}

\usepackage{listings}
\usepackage{color}
\definecolor{dkgreen}{rgb}{0,0.6,0}
\definecolor{gray}{rgb}{0.5,0.5,0.5}
\definecolor{mauve}{rgb}{0.58,0,0.82}
\lstset{ %
  language=Octave,                % the language of the code
  basicstyle=\footnotesize,           % the size of the fonts that are used for the code
  numbers=left,                   % where to put the line-numbers
  numberstyle=\tiny\color{gray},  % the style that is used for the line-numbers
  stepnumber=1,                   % the step between two line-numbers. If it's 1, each 
  %line 
                                  % will be numbered
  numbersep=5pt,                  % how far the line-numbers are from the code
  backgroundcolor=\color{white},      % choose the background color. You must 
  %add \usepackage{color}
  showspaces=false,               % show spaces adding particular underscores
  showstringspaces=false,         % underline spaces within strings
  showtabs=false,                 % show tabs within strings adding particular 
  %underscores
  frame=single,                   % adds a frame around the code
  rulecolor=\color{black},        % if not set, the frame-color may be changed on 
  %line-breaks within not-black text (e.g. commens (green here))
  tabsize=2,                      % sets default tabsize to 2 spaces
  captionpos=b,                   % sets the caption-position to bottom
  breaklines=true,                % sets automatic line breaking
  breakatwhitespace=false,        % sets if automatic breaks should only happen at 
  %whitespace
  title=\lstname,                   % show the filename of files included with 
  %\lstinputlisting;
                                  % also try caption instead of title
  keywordstyle=\color{blue},          % keyword style
  commentstyle=\color{dkgreen},       % comment style
  stringstyle=\color{mauve},         % string literal style
  escapeinside={\%*}{*)},            % if you want to add LaTeX within your code
  morekeywords={*,...}               % if you want to add more keywords to the set
}
  
\begin{document}
	\includepdf{Book-Cover.pdf}   // 是否带封面
%\maketitle

\begin{abstract}
	这里是摘要
\end{abstract}
\noindent\zihao{4}\textbf{关键词}\quad\zihao{-4} 乙肝;预防;;变分法

\tableofcontents %生成目录
\listoffigures% 插图目录
\setcounter{page}{0}
\thispagestyle{empty}
\begin{lstlisting}[title=Myfile, frame=shadowbox]
 numbers=left, 
    numberstyle= \tiny, 
    keywordstyle= \color{ blue!70},
    commentstyle= \color{red!50!green!50!blue!50}, 
    frame=shadowbox, % 阴影效果
    rulesepcolor= \color{ red!20!green!20!blue!20} ,
    escapeinside=``, % 英文分号中可写入中文
    xleftmargin=2em,xrightmargin=2em, aboveskip=1em,
    framexleftmargin=2em
% 代码段
\end{lstlisting}

\begin{figure}[htb]
	\centering
	\includegraphics[width=0.5\textwidth]{fenbu.jpg}
		\caption{我国鼠疫自然疫源地的地理分布}
\label{fig:fenbu}
\end{figure}
  
Paste your own text here and click the 'Check Text' button. Click the colored 
phrases for details on pot ential errors. or use this text too see an few of of the 
problems that LanguageTool can detect. What do you thinks of grammar 
checkers? Please not that they are not perfect. Style issues get a blue marker: It's 
5 P.M. in the afternoon. LanguageTool 3.8 was released on Thursday, 27 June 
2017.
\begin{center}
	\begin{longtable}{|c|c|c|c|}
		\caption{A simple longtable example}\\
		\hline
		\textbf{First entry} & \textbf{Second entry} & \textbf{Third entry} & \textbf{Fourth entry} \\
		\hline
		\endfirsthead
		\multicolumn{4}{c}%
		{\tablename\ \thetable\ -- \textit{Continued from previous page}} \\
		\hline
		\textbf{First entry} & \textbf{Second entry} & \textbf{Third entry} & \textbf{Fourth entry} \\
		\hline
		\endhead
		\hline \multicolumn{4}{r}{\textit{Continued on next page}} \\
		\endfoot
		\hline
		\endlastfoot
		1 & 2 & 3 & 4 \\ 1 & 2 & 3 & 4 \\ 1 & 2 & 3 & 4 \\ 1 & 2 & 3 & 4 \\
		1 & 2 & 3 & 4 \\ 1 & 2 & 3 & 4 \\ 1 & 2 & 3 & 4 \\ 1 & 2 & 3 & 4 \\
		1 & 2 & 3 & 4 \\ 1 & 2 & 3 & 4 \\ 1 & 2 & 3 & 4 \\ 1 & 2 & 3 & 4 \\
		1 & 2 & 3 & 4 \\ 1 & 2 & 3 & 4 \\ 1 & 2 & 3 & 4 \\ 1 & 2 & 3 & 4 \\
		1 & 2 & 3 & 4 \\ 1 & 2 & 3 & 4 \\ 1 & 2 & 3 & 4 \\ 1 & 2 & 3 & 4 \\
		1 & 2 & 3 & 4 \\ 1 & 2 & 3 & 4 \\ 1 & 2 & 3 & 4 \\ 1 & 2 & 3 & 4 \\
		1 & 2 & 3 & 4 \\ 1 & 2 & 3 & 4 \\ 1 & 2 & 3 & 4 \\ 1 & 2 & 3 & 4 \\
		1 & 2 & 3 & 4 \\ 1 & 2 & 3 & 4 \\ 1 & 2 & 3 & 4 \\ 1 & 2 & 3 & 4 \\
		1 & 2 & 3 & 4 \\ 1 & 2 & 3 & 4 \\ 1 & 2 & 3 & 4 \\ 1 & 2 & 3 & 4 \\
		1 & 2 & 3 & 4 \\ 1 & 2 & 3 & 4 \\ 1 & 2 & 3 & 4 \\ 1 & 2 & 3 & 4 \\
		1 & 2 & 3 & 4 \\ 1 & 2 & 3 & 4 \\ 1 & 2 & 3 & 4 \\ 1 & 2 & 3 & 4 \\
		1 & 2 & 3 & 4 \\ 1 & 2 & 3 & 4 \\ 1 & 2 & 3 & 4 \\ 1 & 2 & 3 & 4 \\
	\end{longtable}
\end{center}

\begin{center}
	\begin{longtable}{|p{1cm}|p{2cm}|p{3cm}|p{4cm}|}
		\hline
		\#&指令&描述&使用场景\\ \hline \hline
		\endhead
	\end{longtable}
\end{center}

\begin{center}
	\begin{tabular}{|p{5cm}|p{6cm}|}
		\hline
		\multicolumn{2}{|p{11cm}|}{\textbf{Class  CalendarItem:}该类继承自Calend}\\ \hline
		set(String)& 参数为用户新加入的事项,用于修改text。\textcolor{red}{注意传入的为null,抛出自定义异常}\\ \hline
	\end{tabular}
\end{center}

\newpage
\addcontentsline{toc}{section}{参考文献}
\bibliography{course-template}%记得需要F8
\end{document}